\xname{arch}
\chapter{Architecture of Chord}
\label{chap:arch}

This chapter presents the high-level architecture of Chord, depicted below, and describes its key components.

\begin{center}
\includeimage{0.7}{chord_arch.png}
\end{center}

{\bf Chord Properties}

All inputs to Chord are specified by means of system properties.  Chapter
\ref{chap:properties} describes how to set properties and the meaning of each
property that is recognized by Chord.

The Java program to be analyzed is also specified via properties.  Chapter
\ref{chap:setup} describes how to setup a Java program for analysis using Chord.

{\bf Java Program Representation}

Chord analyzes Java bytecode, not Java source code, and thus only requires the
program's class files.  (Certain analyses, however, present their results at the
Java source code level, and thus require the program's Java source files as
well.) Chord uses the \xlink{Joeq}{http://joeq.sourceforce.net/} Java compiler
framework to convert the Java bytecode of the input Java program, one class file
at a time, into a three-address-like intermediate program representation called
{\it quadcode} that is more suitable for analysis. Chapter \ref{chap:program}
describes the quadcode representation in detail.

{\bf Analysis Scope Construction}

A pre-requisite to analyzing a Java program using any program analysis
framework, including Chord, is to compute the {\it analysis scope}: which parts
of the program to analyze.  Chord implements several standard scope construction
algorithms from the literature that differ in aspects such as scalability,
precision, and usability for the problem at hand.  Chapter \ref{chap:scope}
describes these algorithms in detail.

{\bf Writing and Running Analyses}

Chord provides many standard analyses.  Chapter \ref{chap:predefined} describes
these analyses and explains how to run them.  Moreover, Chord allows users to
write and run their own analyses, possibly atop the provided analyses.

Each analysis in Chord is written modularly, independent of other analyses,
along with lightweight annotations specifying the inputs and outputs of the
analysis.  Chord's runtime automatically computes dependencies between analyses
(e.g., determines which analysis produces as output a result that is needed as
input by another analysis).  Before running a desired analysis, Chord
recursively runs other analyses until the inputs to the desired analysis have
been computed; it finally runs the desired analysis to produce the outputs of
that analysis.

Chord can be invoked in one of two modes: {\it classic} or {\it modern}.  These
two modes defer in the semantics of dependencies between analyses.  In
particular, the classic mode is simpler to understand for novice users (the
dependencies are only data dependencies) but has a sequential runtime, whereas
the modern mode is harder to understand (there are both data and control
dependencies) but has a parallel runtime that is capable of running analyses
without dependencies between them in parallel.  The parallel runtime is based on
\xlink{Habanero-Java}{http://habanero.rice.edu/hj.html}, and the semantics of
the dependencies between analyses is based on the
\xlink{Habanero Concurrent Collections (CnC)}{http://habanero.rice.edu/cnc.html}
declarative parallel programming model.  Chapter \ref{chap:project} expands upon
the modular architecture of analyses in Chord.

Chord provides various {\it analysis templates}: classes containing boilerplate
code that can be extended by users to rapidly prototype different kinds of
analyses.  An example is class RHSAnalysis, named after [Reps, Horowitz, and
Sagiv 1995], which can be extended by users to write a summary-based
inter-procedural context-sensitive static analysis by merely specifying the
abstract domain and intra-procedural transfer functions.  Another example is
DynamicAnalysis, which can be extended by users to write a dynamic analysis by
merely specifying which of various provided events to instrument, and the
transfer functions for those events.  Chapters \ref{chap:writing} and
\ref{chap:running} describe how users can write and run their own analyses in
Chord using the provided analysis templates.

{\bf Dynamic Analysis}

Chord uses the \xlink{Javassist}{http://www.javassist.org/} Java bytecode
manipulation framework for instrumenting bytecode and doing dynamic analysis.
Chord offers the most versatile capabilities of any existing dynamic analysis
framework for Java, particularly the ability to instrument the entire JDK
(including classes in package java.lang).  Specifically, it includes support
for:

\begin{itemize}
\item
offline as well as load-time instrumentation of Java bytecode;
\item
processing of dynamic analysis events online in the same JVM or offline in a
different JVM with an uninstrumented JDK (the latter circumvents performance and
correctness problems that can arise if a single JVM with an instrumented JDK is
used to generate and handle events); and
\item
allowing the event-generating and event-handling JVMs to run either serially
(by storing the entire trace of events to a regular file) or in parallel (by
streaming the trace of events in a piped file).
\end{itemize}

Chapter \ref{chap:dynamic} describes all aspects of dynamic analysis in Chord.

{\bf Datalog Analysis}

A common way to rapidly prototype an analysis in Chord is using a declarative
logic-programming language called Datalog.  Chord uses the BDD-based Datalog
solver \xlink{bddbddb}{http://bddbddb.sourceforge.net/} to run analyes written
in Datalog.  Chapter \ref{chap:datalog} describes all aspects of such analyses.

